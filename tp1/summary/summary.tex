\documentclass{amsart}
\usepackage{amsmath, 
	amssymb,
	amsthm,
	amsfonts, 
	hyperref}

\title{Summary on netlist simulator}
\author{Nguyen Doan Dai}

\begin{document}
	\maketitle
	
	This memo attempt to summarise my implementation of netlist simulator.
	
	\section{Compilation and usage}
	Compilation requires build tool \text{ocamlbuild} of version 0.14.0 or higher. The simulator is implemented in OCaml version 5.1.0. The implementation was compiled and tested in Linux Ubuntu 20.04.6 LTS through Window Subsystem for Linux. The source can be found on \url{https://github.com/enbugging/SystemeNumeriqueENS}.
	
	To compile, use command \texttt{ocamlbuild netlist\_simulator.byte}.
	
	The execution has basic interface, of the form 
	
	\texttt{netlist\_simulator.byte [-print] [-n nr] netfile.net}, 
	
	with
	\begin{itemize}
		\item \texttt{-print}: flag to print only the netlist after scheduling;
		\item \texttt{-n nr}: flag to specify \texttt{nr}, the number of cycles to be simulated;
		\item \texttt{netfile.net}, the \texttt{.net} file containing the description of circuit in netlist. For further information, consult the documentation provided with the package, title \texttt{minijazz.pdf}.
	\end{itemize}
	
	\section{Functionalities}
	As required, it can simulate circuits described in netlist language. I tried to implement the interface as close as possible to the provided prebuilt netlist simulator. Registers, RAM, and ROM are maintained using dictionary, so in particular, there is no hard limit for memory except that allowed for the simulator, at the expense of an additional log factor to the complexity. By default, all variables and memory are initialised to \texttt{0} or \texttt{False}.
	
	This convention does lead to the following behavior. When the scheduler is evoke, for the equation \texttt{RAM addr\_size word\_size read\_addr write\_enable write\_addr write\_data}, we only required that \texttt{read\_addr}, \texttt{write\_enable}, \texttt{write\_addr}, i.e. read address, flag to switch from read to write, and write address, be calculated and well-defined.
	
	In addition to basic functionalities, the simulator is also capable of error handling with detailed messages, including detecting if  
	\begin{itemize}
		\item there is a combinatorial cycle;
		\item there is a variable assigned twice;
		\item there is an unknown variable;
		\item there is an ill-defined value to be written to RAM;
		\item there is an operator with bad arguments;
		\item there is an unknown error making variable not found.
	\end{itemize}
	
	\section{Difficulties}
	
	The main difficulty at the beginning was my unfamiliarity with OCaml, insofar as I was considering re-implement the simulator in an imperative language, such as Python. Another difficult concerned the behavior of memory and registers, which I found baffling and not well-documented enough. No information concerning the initial values for variables (which are required for the implementation of RAM and registers) is provided, and I chose to make a (somewhat arbitrary) convention.
	
\end{document}
